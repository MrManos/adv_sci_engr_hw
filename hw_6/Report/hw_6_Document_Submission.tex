\documentclass[11pt]{article}
\usepackage{subfigure,wrapfig,graphicx,booktabs,fancyhdr,amsmath,amsfonts,appendix,tikz}
\usepackage{bm,amssymb,amsthm,wasysym,color,fullpage,setspace,multirow,placeins}
\usepackage{pgfplots}
\pgfplotsset{compat=1.18}
\usepackage{amsmath,amssymb}
\usepackage{tcolorbox}

% Custom math commands
\newcommand{\vb}{\boldsymbol}
\newcommand{\vbh}[1]{\hat{\boldsymbol{#1}}}
\newcommand{\vbb}[1]{\bar{\boldsymbol{#1}}}
\newcommand{\vbt}[1]{\tilde{\boldsymbol{#1}}}
\newcommand{\vbs}[1]{{\boldsymbol{#1}}^*}
\newcommand{\vbd}[1]{\dot{{\boldsymbol{#1}}}}
\newcommand{\vbdd}[1]{\ddot{{\boldsymbol{#1}}}}
\newcommand{\by}{\times}
\newcommand{\tr}{{\rm tr}}
\newcommand{\cpe}[1]{\left[{#1} \times\right]}
\newcommand{\sfrac}[2]{\textstyle\frac{#1}{#2}}

% Title and Author Information
\title{Homework 6}
\author{Jacob Hands \\ COE 352}
\date{November 13, 2023}

\begin{document}
\maketitle

1) Compute the Jacobian matrix for the following:
\begin{enumerate}
    \item (15pts)
    \[
    \bm{F}(x, y) =
    \begin{bmatrix}
        x^2 + \sin(x) \\
        x(y - 2) \\
        y^2 - 3xy
    \end{bmatrix}
    \]

    \item (15pts)
    \[
    \bm{G}(x, y, z) =
    \begin{bmatrix}
        x^2 - y^2 \\
        3xyz - 5
    \end{bmatrix}
    \]
\end{enumerate}

\subsection*{Solution}
The Jacobian matrix is defined as:
\[
J =
\begin{bmatrix}
    \frac{\partial f_1}{\partial x} & \frac{\partial f_1}{\partial y} & \cdots \\
    \frac{\partial f_2}{\partial x} & \frac{\partial f_2}{\partial y} & \cdots \\
    \frac{\partial f_3}{\partial x} & \frac{\partial f_3}{\partial y} & \cdots \\
\end{bmatrix}
\]

\subsubsection*{1) For \(\bm{F}(x, y)\):}
\[
J_{\bm{F}}(x, y) =
\begin{bmatrix}
    2x + \cos(x) & 0 \\
    y - 2 & x  \\
    -3y & 2y - 3x 
\end{bmatrix}
\]

\subsubsection*{2) For \(\bm{G}(x, y, z)\):}
\[
J_{\bm{G}}(x, y, z) =
\begin{bmatrix}
    2x & -2y & 0 \\
    3yz & 3xz & 3xy
\end{bmatrix}
\]

\section*{Problem 2: Jacobian of Polar Coordinates Transformation}

\subsection*{Given:}
\[
x = r \cos \theta, \quad y = r \sin \theta
\]

The transformation can be expressed as:
\[
T(r, \theta) =
\begin{cases}
x = r \cos \theta \\
y = r \sin \theta
\end{cases}
\]

\subsection*{Jacobian Matrix:}
The Jacobian matrix is defined as:
\[
\frac{\partial(x, y)}{\partial(r, \theta)} =
\begin{bmatrix}
\frac{\partial x}{\partial r} & \frac{\partial x}{\partial \theta} \\
\frac{\partial y}{\partial r} & \frac{\partial y}{\partial \theta}
\end{bmatrix}
=
\begin{bmatrix}
\cos \theta & -r \sin \theta \\
\sin \theta & r \cos \theta
\end{bmatrix}
\]

\subsection*{Jacobian Determinant:}
\[
\text{det}\left( \frac{\partial(x, y)}{\partial(r, \theta)} \right) =
\begin{vmatrix}
\cos \theta & -r \sin \theta \\
\sin \theta & r \cos \theta
\end{vmatrix}
= (\cos \theta)(r \cos \theta) - (-r \sin \theta)(\sin \theta)
\]
\[
= r \cos^2 \theta + r \sin^2 \theta
\]
\[
= r(\cos^2 \theta + \sin^2 \theta)
\]
\[
= r
\]

\subsection*{Conclusion:}
The Jacobian determinant for the polar coordinate transformation is:
\[
\boxed{r}
\]

\section*{Problem 3: Derive the Weak Form of the 1D Wave Equation}

\subsection*{Given:}
The 1D wave equation is:
\[
\frac{\partial^2 u(x, t)}{\partial t^2} - c^2 \nabla^2 u(x, t) = f(x, t)
\]
where \(c\) is a scalar-valued wave speed.

\subsection*{Steps:}
To derive the weak form:
\begin{itemize}
    \item Multiply by a test function \(V(x)\).
    \item Integrate by parts (IBP) for spatial derivatives.
\end{itemize}

\subsubsection*{We Know:}
The term \(\frac{\partial^2 u(x, t)}{\partial t^2}\) is a temporal derivative because \(\frac{\partial u(x, t)}{\partial t}\) is with respect to time.

\subsection*{Start with the Weak Form:}
\[
\int V(x) \left( \frac{\partial^2 u(x, t)}{\partial t^2} - c^2 \nabla^2 u(x, t) \right) dx = \int f(x, t) V(x) dx
\]

Split into two terms:
\[
\int \frac{\partial^2 u(x, t)}{\partial t^2} V(x) dx - c^2 \int \nabla^2 u(x, t) V(x) dx = \int f(x, t) V(x) dx
\]

Using integration by parts (IBP):
\[
V(x) \cdot \nabla u(x, t) - \int \nabla u(x, t) \cdot \nabla V(x) dx
\]

where:
\[
u = V(x), \quad  du = \frac{dV(x)}{dx} \quad v = \frac{\partial u(x, t)}{\partial x}, \quad dv = \frac{\partial^2 u(x, t)}{\partial x^2} dx, \quad \frac{d}{dx} = \nabla
\]

Substituting, we obtain:
\[
\int \frac{\partial^2 u(x, t)}{\partial t^2} V(x) dx - c^2 \int V(x) \cdot \nabla u(x, t) - \int \nabla u(x, t) \cdot \nabla V(x) dx = \int f(x, t) V(x) dx
\]

\subsection*{Final Weak Form:}
\[
\boxed{\int \frac{\partial^2 u(x, t)}{\partial t^2} V(x) dx - c^2 \left[ \int V(x) \cdot \nabla u(x, t) dx - \int \nabla u(x, t) \cdot \nabla V(x) dx \right] = \int f(x, t) V(x) dx}
\]

\section*{Problem 4: Lagrangian Interpolation Approximation}

\subsection*{Given:}
Let \(f(x) = e^x\) for \(0 \leq x \leq 2\). Approximate \(f(0.25)\) using Lagrangian linear interpolation with \(x_0 = 0\) and \(x_1 = 0.5\). Compare your answer to the real one.

\subsection*{Steps:}
We know that Lagrangian polynomials have:
\[
\frac{d^2}{dx^2} = 0 \quad \text{at desired points.}
\]

The given points are:
\[
f(0) = e^0 = 1, \quad f(0.5) = e^{0.5} \approx 1.6487
\]

The Lagrangian interpolation formula is:
\[
f(x) = y_0 \frac{(x - x_1)}{(x_0 - x_1)} + y_1 \frac{(x - x_0)}{(x_1 - x_0)}
\]
Substituting the values:
\[
f(x) = 1 \cdot \frac{(x - 0.5)}{(0 - 0.5)} + 1.6487 \cdot \frac{(x - 0)}{(0.5 - 0)}
\]

Simplify:
\[
f(x) = -2x + 1 + 1.6487(2x)
\]
\[
f(x) = 1.2974x + 1
\]

\subsection*{Approximation for \(f(0.25)\):}
Substitute \(x = 0.25\):
\[
f(0.25) = 1.2974(0.25) + 1 = 1.32435
\]

\subsection*{Exact Value:}
The exact value is:
\[
f(0.25) = e^{0.25} \approx 1.284025
\]

\subsection*{Percentage Error:}
\[
\text{Error} = \frac{|1.32435 - 1.284025|}{1.284025} \approx 0.0314 = 3.14\%
\]

\subsection*{Conclusion:}
The approximation \(f(0.25) = 1.32435\) is about \(3.14\%\) above the exact value \(1.284025\).

\section*{Problem 5: Lagrangian Interpolating Polynomial}

\subsection*{Given:}
Let \(P_3(x)\) be the Lagrangian interpolating polynomial for the data \((0, 0)\), \((0.5, y)\), \((1, 3)\), and \((2, 2)\). Find \(y\) if the coefficient of the \(x^3\) term in \(P_3(x)\) is \(6\).

\subsection*{Steps:}
The Lagrangian polynomial is expressed as:
\[
f(x) = f_1(x) + f_2(x) + f_3(x) + f_4(x)
\]

Each term \(f_k(x)\) is:
\[
f_k(x) = y_k \prod_{\substack{i=0 \\ i \neq k}}^{n} \frac{x - x_i}{x_k - x_i}
\]

For the given points:
\[
f(x) = 0 \cdot \frac{(x - 0.5)(x - 1)(x - 2)}{(0 - 0.5)(0 - 1)(0 - 2)} 
+ y \cdot \frac{(x - 0)(x - 1)(x - 2)}{(0.5 - 0)(0.5 - 1)(0.5 - 2)}
+ 3 \cdot \frac{(x - 0)(x - 0.5)(x - 2)}{(1 - 0)(1 - 0.5)(1 - 2)}
+ 2 \cdot \frac{(x - 0)(x - 0.5)(x - 1)}{(2 - 0)(2 - 0.5)(2 - 1)}
\]

Simplify each term:
\[
f(x) = y \cdot \frac{(x - 0)(x - 1)(x - 2)}{(0.5)(-0.5)(-1.5)}
+ 3 \cdot \frac{(x - 0)(x - 0.5)(x - 2)}{(1)(0.5)(-1)}
+ 2 \cdot \frac{(x - 0)(x - 0.5)(x - 1)}{(2)(1.5)(1)}
\]

After simplifying:
\[
f(x) = \frac{1.5y x^3 - 15x^2 + 47.8125x - 32.8125}{2.8125}
\]

\subsection*{Finding \(y\):}
The coefficient of \(x^3\) is given as \(6\). Thus:
\[
\frac{1.5y}{2.8125} = 6
\]

Solve for \(y\):
\[
1.5y = 6 \cdot 2.8125 = 16.875
\]
\[
y = \frac{16.875}{1.5} \approx 21.25
\]

\subsection*{Conclusion:}
The value of \(y\) is:
\[
\boxed{21.25}
\]

\end{document}
