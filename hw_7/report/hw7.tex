\documentclass[11pt]{article}
\usepackage{subfigure,wrapfig,graphicx,booktabs,fancyhdr,amsmath,amsfonts,appendix,tikz}
\usepackage{bm,amssymb,amsthm,wasysym,color,fullpage,setspace,multirow,placeins}
\usepackage{pgfplots}
\pgfplotsset{compat=1.18}
\usepackage{amsmath,amssymb}
\usepackage{tcolorbox}

% Custom math commands
\newcommand{\vb}{\boldsymbol}
\newcommand{\vbh}[1]{\hat{\boldsymbol{#1}}}
\newcommand{\vbb}[1]{\bar{\boldsymbol{#1}}}
\newcommand{\vbt}[1]{\tilde{\boldsymbol{#1}}}
\newcommand{\vbs}[1]{{\boldsymbol{#1}}^*}
\newcommand{\vbd}[1]{\dot{{\boldsymbol{#1}}}}
\newcommand{\vbdd}[1]{\ddot{{\boldsymbol{#1}}}}
\newcommand{\by}{\times}
\newcommand{\tr}{{\rm tr}}
\newcommand{\cpe}[1]{\left[{#1} \times\right]}
\newcommand{\sfrac}[2]{\textstyle\frac{#1}{#2}}

% Title and Author Information
\title{Homework 7}
\author{Jacob Hands \\ COE 352}
\date{December 4, 2023}

\begin{document}
\maketitle

\section*{1. (20pts) Approximate $\int_0^\pi \sin x \, dx$ using the 4-point quadrature rule on a parent domain of $-1 \leq \xi \leq 1$.}

\textbf{Solution:}

From the notes, we use the point weights and locations:
\[
N_q = 4, \quad q_i: \{-0.861, -0.348, 0.348, 0.861\}, \quad w_i: \{0.348, 0.652, 0.652, 0.348\}.
\]

The integral is transformed using element mapping because the grid is unstructured:
\[
\int_a^b f(x) \, dx = \int_{-1}^1 f(x(\xi)) \frac{dx}{d\xi} \, d\xi = \sum_{i=1}^{N_q} w_i f(x(q_i)).
\]

Mapping transformation:
\[
x(\xi) = \frac{\pi}{2} \frac{\xi + 1}{2}, \quad \frac{dx}{d\xi} = \frac{\pi}{2}.
\]

The quadrature formula becomes:
\[
\sum_{i=1}^{N_q} w_i f(x_i) \frac{\pi}{2}.
\]

Evaluations:
\[
f(x_1) = \sin\left(\frac{\pi}{2}(-0.861 + 1)\right) = 0.34025,
\]
\[
f(x_2) = \sin\left(\frac{\pi}{2}(-0.348 + 1)\right) = 1.34190,
\]
\[
f(x_3) = \sin\left(\frac{\pi}{2}(0.348 + 1)\right) = 1.34190,
\]
\[
f(x_4) = \sin\left(\frac{\pi}{2}(0.861 + 1)\right) = 0.34025.
\]

Result:
\[
\int_0^\pi \sin x \, dx \approx \frac{\pi}{2} \left[ (0.348)(0.34025) + (0.652)(1.34190) + (0.652)(1.34190) + (0.348)(0.34025) \right] \approx 1.984.
\]

\section*{2. (20pts) Find the constants $c_0$, $c_1$, and $x_1$ so that the quadrature formula $\int_0^1 f(x) \, dx = c_0 f(0) + c_1 f(x_1)$ has the highest possible degree of precision.}

\textbf{Solution:}

For highest precision, we solve:
\[
\int_0^1 1 \, dx = c_0 + c_1 = 1,
\]
\[
\int_0^1 x \, dx = c_1 x_1 = \frac{1}{2},
\]
\[
\int_0^1 x^2 \, dx = c_1 x_1^2 = \frac{1}{3}.
\]

Solving step-by-step:
\[
c_1 x_1 = \frac{1}{2}, \quad x_1 = \frac{2}{3}, \quad c_1 = \frac{3}{2}.
\]

\[
c_0 = 1 - c_1 = 1 - \frac{3}{2} = \frac{1}{4}.
\]

Final values:
\[
c_0 = \frac{1}{4}, \quad c_1 = \frac{3}{2}, \quad x_1 = \frac{2}{3}.
\]

\section*{3. (20pts) Using 2D Gaussian quadrature, compute the integral of $f(x, y) = x^2y^2$ on $[-1, 1] \times [-1, 1]$.}

\textbf{Solution:}

The 2D integral:
\[
\int_{-1}^1 \int_{-1}^1 f(x, y) \, dx \, dy \approx \sum_{i=1}^2 \sum_{j=1}^2 w_i w_j f(x_i, y_j),
\]
where \(N = 2\), \(q_i = \{-0.58, 0.58\}, w_i = \{1, 1\}\).

Substitute:
\[
f(-0.58, -0.58) = (-0.58)^2(-0.58)^2, \quad f(-0.58, 0.58) = (-0.58)^2(0.58)^2,
\]
\[
f(0.58, -0.58) = (0.58)^2(-0.58)^2, \quad f(0.58, 0.58) = (0.58)^2(0.58)^2.
\]

Compute:
\[
\int_{-1}^1 \int_{-1}^1 f(x, y) \, dx \, dy \approx 4 \times (0.58^2)^2 = 0.444.
\]

\section*{4. (20pts) Define the 2D linear Lagrange polynomials and integrate $f(x, y) = \frac{1}{4}(1 - x - y + x^2 y^2)$.}

\textbf{Solution:}

The 2D Lagrange polynomials:
\[
\phi_1(x, y) = \frac{(1-x)(1-y)}{4}, \quad \phi_2(x, y) = \frac{(1-x)(1+y)}{4},
\]
\[
\phi_3(x, y) = \frac{(1+x)(1-y)}{4}, \quad \phi_4(x, y) = \frac{(1+x)(1+y)}{4}.
\]

Interpolation:
\[
f(x, y) \approx \phi_1(x, y) f(-1, -1) + \phi_2(x, y) f(-1, 1) + \phi_3(x, y) f(1, -1) + \phi_4(x, y) f(1, 1).
\]

Substitute and simplify:
\[
f(x, y) = \frac{1}{4}(1 - x - y + x^2 y^2).
\]

Integration:
\[
\int_{-1}^1 \int_{-1}^1 f(x, y) \, dx \, dy = \int_0^1 \int_0^1 \left(1 - \frac{x}{4}\right) dx \, dy = 1.
\]

\section*{5. (20pts) Find the Jacobian matrix.}

Find the Jacobian matrix:
\[
\mathbf{J} =
\begin{pmatrix}
\frac{\partial x}{\partial \xi} & \frac{\partial y}{\partial \xi} \\
\frac{\partial x}{\partial \eta} & \frac{\partial y}{\partial \eta}
\end{pmatrix},
\]
using 2D Lagrange interpolation mapping from a quadrilateral with the nodal coordinates \((0,0), (1,0), (2,2), (0,1)\).

\textbf{Step 1: Define the 2D Lagrange polynomials}

\[
x(\xi, \eta) = \sum_{i=1}^4 N_i(\xi, \eta) x_i, \quad y(\xi, \eta) = \sum_{i=1}^4 N_i(\xi, \eta) y_i.
\]

The shape functions \(N_i(\xi, \eta)\) are:
\[
N_1 = \frac{1}{4}(1-\xi)(1-\eta), \quad N_2 = \frac{1}{4}(1+\xi)(1-\eta),
\]
\[
N_3 = \frac{1}{4}(1+\xi)(1+\eta), \quad N_4 = \frac{1}{4}(1-\xi)(1+\eta).
\]

\textbf{Step 2: Compute partial derivatives of \(N_i\):}

\[
\frac{\partial N_1}{\partial \xi} = -\frac{1}{4}(1-\eta), \quad \frac{\partial N_1}{\partial \eta} = -\frac{1}{4}(1-\xi),
\]
\[
\frac{\partial N_2}{\partial \xi} = \frac{1}{4}(1-\eta), \quad \frac{\partial N_2}{\partial \eta} = -\frac{1}{4}(1+\xi),
\]
\[
\frac{\partial N_3}{\partial \xi} = \frac{1}{4}(1+\eta), \quad \frac{\partial N_3}{\partial \eta} = \frac{1}{4}(1+\xi),
\]
\[
\frac{\partial N_4}{\partial \xi} = -\frac{1}{4}(1+\eta), \quad \frac{\partial N_4}{\partial \eta} = \frac{1}{4}(1-\xi).
\]

\textbf{Step 3: Compute the Jacobian entries}

For \(x(\xi, \eta)\) and \(y(\xi, \eta)\):
\[
x_1 = 0, \; x_2 = 1, \; x_3 = 2, \; x_4 = 0,
\]
\[
y_1 = 0, \; y_2 = 0, \; y_3 = 2, \; y_4 = 1.
\]

\[
\frac{\partial x}{\partial \xi} = \frac{\partial N_1}{\partial \xi} x_1 + \frac{\partial N_2}{\partial \xi} x_2 + \frac{\partial N_3}{\partial \xi} x_3 + \frac{\partial N_4}{\partial \xi} x_4,
\]
\[
\frac{\partial x}{\partial \xi} = -\frac{1}{4}(1-\eta)(0) + \frac{1}{4}(1-\eta)(1) + \frac{1}{4}(1+\eta)(2) + -\frac{1}{4}(1+\eta)(0),
\]
\[
\frac{\partial x}{\partial \xi} = \frac{1}{4}(1-\eta) + \frac{1}{2}(1+\eta) = \frac{1}{4}(\eta + 3).
\]

Similarly:
\[
\frac{\partial y}{\partial \xi} = \frac{\partial N_1}{\partial \xi} y_1 + \frac{\partial N_2}{\partial \xi} y_2 + \frac{\partial N_3}{\partial \xi} y_3 + \frac{\partial N_4}{\partial \xi} y_4,
\]
\[
\frac{\partial y}{\partial \xi} = -\frac{1}{4}(1-\eta)(0) + \frac{1}{4}(1-\eta)(0) + \frac{1}{4}(1+\eta)(2) + -\frac{1}{4}(1+\eta)(1),
\]
\[
\frac{\partial y}{\partial \xi} = \frac{1}{4}(1+\eta).
\]

For \(\frac{\partial x}{\partial \eta}\):
\[
\frac{\partial x}{\partial \eta} = \frac{\partial N_1}{\partial \eta} x_1 + \frac{\partial N_2}{\partial \eta} x_2 + \frac{\partial N_3}{\partial \eta} x_3 + \frac{\partial N_4}{\partial \eta} x_4,
\]
\[
\frac{\partial x}{\partial \eta} = -\frac{1}{4}(1-\xi)(0) + -\frac{1}{4}(1+\xi)(1) + \frac{1}{4}(1+\xi)(2) + \frac{1}{4}(1-\xi)(0),
\]
\[
\frac{\partial x}{\partial \eta} = \frac{1}{4}(1+\xi).
\]

For \(\frac{\partial y}{\partial \eta}\):
\[
\frac{\partial y}{\partial \eta} = \frac{\partial N_1}{\partial \eta} y_1 + \frac{\partial N_2}{\partial \eta} y_2 + \frac{\partial N_3}{\partial \eta} y_3 + \frac{\partial N_4}{\partial \eta} y_4,
\]
\[
\frac{\partial y}{\partial \eta} = -\frac{1}{4}(1-\xi)(0) + -\frac{1}{4}(1+\xi)(0) + \frac{1}{4}(1+\xi)(2) + \frac{1}{4}(1-\xi)(1),
\]
\[
\frac{\partial y}{\partial \eta} = \frac{1}{4}(1+\xi).
\]

\textbf{Final Jacobian Matrix:}
\[
\mathbf{J} =
\begin{pmatrix}
\frac{1}{4}(\eta + 3) & \frac{1}{4}(1+\eta) \\
\frac{1}{4}(1+\xi) & \frac{1}{4}(1+\xi)
\end{pmatrix}.
\]

\end{document}
